In this thesis we study two important problems in computational biology, one pertaining to primary analysis of sequencing data, and the second pertaining to secondary analysis of sequences to obtain biological insights using machine-learning.
Sequence alignment is one of the most powerful and important tools in the field of computational biology. Read alignment is often the first step in many analyses like structural variant detection, genome assembly or variant calling. Long read sequencing technologies have improved the quality of results across all these analyses. They remain, however, plagued by sequencing errors and pose algorithmic challenges to alignment. A prevalent technique to reduce the detrimental effects of these errors is homopolymer compression, which targets the most common type of long-read sequencing error. 
We present a more general framework than homopolymer compression, which we call mapping-friendly sequence reductions (MSR). We then show that some of these MSRs improve the accuracy of read alignments across whole human, \textit{drosophila} and \textit{E. coli} genomes. 
Improvements in sequence alignment methods are crucial for downstream analyses. For instance, multiple sequence alignments are indispensable when studying resistance in viruses. With the ever growing quantity of annotated, high-quality multiple sequence alignments it has become possible and useful to study drug resistance in viruses with machine learning methods. 
We used a very large multiple sequence alignment of British HIV sequences to train multiple classifiers to discriminate between treatment-naive and treatment-experienced sequences. By studying important classifier features we identified resistance-associated mutations. We then removed known drug resistance associated signal from the data before training, keeping classifying power, and identified 6 novel resistance associated mutations. Further study indicated that these were most likely accessory in nature and linked to known resistance mutations.
